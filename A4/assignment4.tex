\documentclass[12pt,letterpaper]{article}
\usepackage{fullpage}
\usepackage[top=2cm, bottom=4.5cm, left=2.5cm, right=2.5cm]{geometry}
\usepackage{amsmath,amsthm,amsfonts,amssymb,amscd}
\usepackage{lastpage}
\usepackage{enumerate}
\usepackage{fancyhdr}
\usepackage{mathrsfs}
\usepackage{xcolor}
\usepackage{graphicx}
\usepackage{listings}
\usepackage{hyperref}

\hypersetup{%
	colorlinks=true,
	linkcolor=blue,
	linkbordercolor={0 0 1}
}

\renewcommand\lstlistingname{Algorithm}
\renewcommand\lstlistlistingname{Algorithms}
\def\lstlistingautorefname{Alg.}

\lstdefinestyle{Python}{
	language        = Python,
	frame           = lines, 
	basicstyle      = \footnotesize,
	keywordstyle    = \color{blue},
	stringstyle     = \color{green},
	commentstyle    = \color{red}\ttfamily
}

\setlength{\parindent}{0.0in}
\setlength{\parskip}{0.05in}

% Edit these as appropriate
\newcommand\course{CS 320}
\newcommand\hwnumber{4}                  % <-- homework number
\newcommand\NetIDa{Edwards, Ryan}           % <-- NetID of person #1
\newcommand\NetIDb{399067099}           % <-- NetID of person #2 (Comment this line out for problem sets)

\pagestyle{fancyplain}
\headheight 35pt
\lhead{\NetIDa}
\lhead{\NetIDa\\\NetIDb}                 % <-- Comment this line out for problem sets (make sure you are person #1)
\chead{\textbf{\Large Assignment \hwnumber}}
\rhead{\course \\ \today}
\lfoot{}
\cfoot{}
\rfoot{\small\thepage}
\headsep 1.5em

\begin{document}
	\section*{Theoretical questions}

	
	\begin{enumerate}
		\item 
		Illustrate that if a new email from the address $a \in A$, it always gets through (1 pts).
		
		\begin{proof}
		Consider the email address: redw764@aucklanduni.ac.nz
		
		In binary, this is equivalent to:
		
		01110010 01100101 01100100 01110111 00110111 00110110 00110100 01000000 01100001 01110101 01100011 01101011 01101100 01100001 01101110 01100100 01110101 01101110 01101001 00101110 01100001 01100011 00101110 01101110 01111010
		
		Let this be represented as an integer $x$ such that $x \in A$
		
		When the hash table is constructed, the function will set $B[h(x)] = 1$
		
		When an email is received, the email filter will apply the hash function $h(x)$
		
		By definition:
		\begin{itemize}
			\item 
			If $B[h(x)] = 1$ the email will go through
			\item 
			If $B[h(x)] = 0$ the email is considered spam
		\end{itemize}
		
		Because $B[h(x)] = 1$ was set when the hash table was constructed, and because a hash function will always return the same output for $h(x)$, the lookup function will return $1$ and the email will go through.
		
	\end{proof}
		
		\item
		Given any position $0 \leq i < n$, what is the probability that $B[i] = 1$ (2 pts).
		
		As there is a universal hashing function for integers
		
		$$ h_{ab}(x) = ((ax + b) \mod p) \mod n $$
		
		As $n = 8,000,000,000$ the probability of a collision is $Pr_h[h(x)=h(y)] \leq \frac{1}{n} \leq \frac{1}{8\text{B}}$
		
		Therefore, the probability of $B[i] = 0$ after $1$ insert is $\geq 1 - \frac{1}{8\text{B}}$
		
		The probability of $B[i] = 0$ after $1\text{B}$ inserts is $\geq (1 - \frac{1}{8\text{B}})^{1\text{B}}$		
		
		Thus, the the probability that $B[i] = 1$ is simply
		
		$$\leq 1 - (1 - \frac{1}{8\text{B}})^{1\text{B}} \lessapprox 0.11750 $$
		
		\item 
		Given a spam email from the address $a' \notin A$, what is the probability that it gets through (2 pts)
		
		As, by definition, a universal hashing function will uniformly distribute a new hashed value, the probability of collision is the probability that $B[i] = 1$.
		
		Thus, the probability of $B[h(a')] = 1$ is $\lessapprox 0.11750$
		
		\pagebreak
		
		\section*{Practical implementation}
		
		\begin{lstlisting}[style = Python]	
	import random

        def main(): 
            # Initial Values
            n = 8000000
            p = 8024047 # prime number > n

            # Initalise an empty Hash Table
            hash_table = [0] * n

            # 0 <= a, b < p
            a = random.randrange(0, p)
            b = random.randrange(0, p)
            
            # Create the email address list
            total_addresses = 1000000
            email_address_list = [i for i in range(1, total_addresses + 1)]
            
            for address in email_address_list:
                hash_table[universal_hash(a, b, n, p, address)] = 1 


        ### Question one ###
            try:
                for number in email_address_list:
                    hash_value = universal_hash(a , b, n, p, number)
                    
                    if hash_table[hash_value] == 0:
                        raise SpamDetected

            except SpamDetected:
                print("Spam test failed")

            else:
                print("Spam test passed")

        ### Question two ###
            # Based on formula in theoretical question 2
            theoretical_probability = 1 - (1 - 1/n) ** total_addresses

            print("Theoretical Probability =", theoretical_probability)
            
        \end{lstlisting}
        \pagebreak
        \begin{lstlisting}[style = Python]	
        ### Question three ###
            spam_email_count = 0
            spam_email_no = 1000
            
            for i in range(spam_email_no):
                random_address = random.randrange(total_addresses + 1, 9999999)
                hash_value = universal_hash(a, b, n, p, random_address)
                if hash_table[hash_value] == 1:
                    spam_email_count += 1

            print("Simulated Probability =", spam_email_count / spam_email_no)
            print("No. Unblocked Spam =", spam_email_count)
            print()


        # Hashing function based on universal hash family for integer
        def universal_hash(a, b, n,  p, x):
            return ((a * x + b) % p) % n

        class Error(Exception):
           """Base class for other exceptions"""
           pass
        class SpamDetected(Error):
           """Spam has been detected"""
           pass

        main()
		
		\end{lstlisting}
		
		Sample Output:
		
		\includegraphics[width=0.65\linewidth]{output.png}
		
		
	\end{enumerate}
	
\end{document}